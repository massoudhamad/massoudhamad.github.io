\documentclass[12pt,a4paper]{article}
\usepackage[utf8]{inputenc}
\usepackage[T1]{fontenc}
\usepackage[margin=2.5cm]{geometry}
\usepackage{graphicx}
\usepackage{hyperref}
\usepackage{enumitem}
\usepackage{xcolor}
\usepackage{fancyhdr}
\usepackage{titlesec}
\usepackage{listings}

\definecolor{primary}{RGB}{220, 38, 38}
\definecolor{codebg}{RGB}{248, 249, 250}

\lstset{
    backgroundcolor=\color{codebg},
    basicstyle=\ttfamily\small,
    breaklines=true,
    frame=single
}

\hypersetup{
    colorlinks=true,
    linkcolor=primary,
    urlcolor=primary
}

\pagestyle{fancy}
\fancyhf{}
\fancyhead[L]{Software Development Project}
\fancyhead[R]{Lecture 7: Requirements}
\fancyfoot[C]{\thepage}

\title{\textbf{Lecture 7: Requirements Gathering and Analysis}\\[0.5cm]\large From User Needs to Software Specifications}
\author{State University of Zanzibar (SUZA)\\BSc Computer Science}
\date{}

\begin{document}

\maketitle
\tableofcontents
\newpage

\section{Introduction to Requirements Engineering}

\subsection{What are Requirements?}
Requirements are statements that describe what a software system must do and how it must perform. They form the foundation of any software project.

\textbf{Why Requirements Matter:}
\begin{itemize}
    \item 40-60\% of project failures are due to poor requirements
    \item Fixing requirements errors late in development costs 100x more
    \item Clear requirements lead to better estimates and planning
    \item Requirements are the contract between developers and stakeholders
\end{itemize}

\subsection{Types of Requirements}

\subsubsection{Functional Requirements}
Describe \textbf{what} the system should do.

\textbf{Examples:}
\begin{itemize}
    \item The system shall allow users to register with email and password
    \item The system shall send email notifications when orders are placed
    \item Users shall be able to search products by name or category
    \item The admin shall be able to generate monthly sales reports
\end{itemize}

\subsubsection{Non-Functional Requirements}
Describe \textbf{how} the system should perform.

\textbf{Categories:}
\begin{itemize}
    \item \textbf{Performance:} Response time, throughput, capacity
    \item \textbf{Security:} Authentication, authorization, encryption
    \item \textbf{Usability:} Ease of use, accessibility, learning curve
    \item \textbf{Reliability:} Uptime, fault tolerance, recovery
    \item \textbf{Scalability:} Handle growth in users and data
    \item \textbf{Maintainability:} Ease of updates and fixes
\end{itemize}

\textbf{Examples:}
\begin{itemize}
    \item The system shall respond to user requests within 3 seconds
    \item The system shall be available 99.9\% of the time
    \item The system shall support 1000 concurrent users
    \item All passwords shall be encrypted using bcrypt
\end{itemize}

\section{Requirements Gathering Techniques}

\subsection{1. Stakeholder Interviews}

\textbf{What:} One-on-one conversations with key stakeholders

\textbf{How to Conduct:}
\begin{enumerate}
    \item Prepare questions in advance
    \item Start with open-ended questions
    \item Listen actively and take notes
    \item Ask follow-up questions
    \item Summarize and confirm understanding
\end{enumerate}

\textbf{Sample Questions:}
\begin{itemize}
    \item What problem are you trying to solve?
    \item Who will use this system?
    \item What tasks do you do daily that this system should support?
    \item What are your biggest pain points with the current process?
    \item What would success look like for this project?
\end{itemize}

\subsection{2. Questionnaires and Surveys}

\textbf{Best For:} Gathering input from many users

\textbf{Tips:}
\begin{itemize}
    \item Keep questions clear and concise
    \item Use a mix of open and closed questions
    \item Include rating scales (1-5)
    \item Test the survey before distributing
\end{itemize}

\subsection{3. Observation}

\textbf{What:} Watch users perform their current tasks

\textbf{Benefits:}
\begin{itemize}
    \item Understand actual workflow (not just described)
    \item Identify pain points users don't mention
    \item See workarounds and inefficiencies
\end{itemize}

\subsection{4. Document Analysis}

\textbf{What:} Review existing documentation

\textbf{Documents to Review:}
\begin{itemize}
    \item Current system documentation
    \item Business process documents
    \item Reports and forms
    \item Competitor products
\end{itemize}

\subsection{5. Workshops and Brainstorming}

\textbf{What:} Group sessions with stakeholders

\textbf{Techniques:}
\begin{itemize}
    \item Brainstorming sessions
    \item JAD (Joint Application Development)
    \item Focus groups
    \item Prototyping sessions
\end{itemize}

\section{User Stories}

\subsection{What is a User Story?}
A user story is a short, simple description of a feature from the user's perspective.

\subsection{User Story Format}
\begin{center}
\fbox{\parbox{0.8\textwidth}{
\textbf{As a} [type of user],\\
\textbf{I want} [some goal/action],\\
\textbf{So that} [benefit/reason].
}}
\end{center}

\subsection{Examples}

\textbf{E-commerce System:}
\begin{itemize}
    \item As a \textbf{customer}, I want to \textbf{search for products by name} so that I can \textbf{quickly find what I'm looking for}.
    \item As a \textbf{customer}, I want to \textbf{add items to my cart} so that I can \textbf{purchase multiple items at once}.
    \item As an \textbf{admin}, I want to \textbf{view sales reports} so that I can \textbf{track business performance}.
\end{itemize}

\textbf{Student Management System:}
\begin{itemize}
    \item As a \textbf{student}, I want to \textbf{view my grades online} so that I can \textbf{track my academic progress}.
    \item As a \textbf{lecturer}, I want to \textbf{upload course materials} so that \textbf{students can access them anytime}.
    \item As a \textbf{registrar}, I want to \textbf{generate transcripts} so that \textbf{students can apply for jobs or further studies}.
\end{itemize}

\subsection{INVEST Criteria for Good User Stories}

\begin{itemize}
    \item \textbf{I}ndependent - Can be developed separately
    \item \textbf{N}egotiable - Details can be discussed
    \item \textbf{V}aluable - Provides value to user
    \item \textbf{E}stimable - Can estimate effort
    \item \textbf{S}mall - Fits in one sprint
    \item \textbf{T}estable - Can verify completion
\end{itemize}

\subsection{Acceptance Criteria}

Each user story should have acceptance criteria that define when it's complete.

\textbf{Format: Given-When-Then}
\begin{lstlisting}
Given [some context/precondition]
When [action is performed]
Then [expected result]
\end{lstlisting}

\textbf{Example:}
\begin{lstlisting}
User Story: As a customer, I want to login
so that I can access my account.

Acceptance Criteria:
1. Given I am on the login page
   When I enter valid credentials
   Then I am redirected to my dashboard

2. Given I am on the login page
   When I enter invalid credentials
   Then I see an error message

3. Given I am logged in
   When I am inactive for 30 minutes
   Then I am automatically logged out
\end{lstlisting}

\section{Use Cases}

\subsection{What is a Use Case?}
A use case describes a specific interaction between a user (actor) and the system to achieve a goal.

\subsection{Use Case Components}

\begin{tabular}{|l|p{10cm}|}
\hline
\textbf{Component} & \textbf{Description} \\
\hline
Use Case ID & Unique identifier (e.g., UC-001) \\
\hline
Name & Descriptive name (e.g., "User Login") \\
\hline
Actor(s) & Who initiates the use case \\
\hline
Description & Brief summary \\
\hline
Preconditions & What must be true before \\
\hline
Main Flow & Step-by-step normal scenario \\
\hline
Alternative Flows & Variations from main flow \\
\hline
Exception Flows & Error handling \\
\hline
Postconditions & What is true after \\
\hline
\end{tabular}

\subsection{Example Use Case}

\begin{tabular}{|l|p{10cm}|}
\hline
\textbf{Use Case ID} & UC-001 \\
\hline
\textbf{Name} & User Registration \\
\hline
\textbf{Actor} & New User \\
\hline
\textbf{Description} & User creates a new account \\
\hline
\textbf{Preconditions} & User is not logged in \\
\hline
\textbf{Main Flow} &
1. User clicks "Register"\\
2. System displays registration form\\
3. User enters name, email, password\\
4. User clicks "Submit"\\
5. System validates input\\
6. System creates account\\
7. System sends confirmation email\\
8. System displays success message \\
\hline
\textbf{Alternative Flow} &
5a. Email already exists:\\
\quad System displays error message\\
\quad User can try different email \\
\hline
\textbf{Postconditions} & User account is created and active \\
\hline
\end{tabular}

\section{Requirements Prioritization}

\subsection{MoSCoW Method}

\begin{itemize}
    \item \textbf{Must Have (M):} Critical for launch, non-negotiable
    \item \textbf{Should Have (S):} Important but not critical
    \item \textbf{Could Have (C):} Nice to have if time permits
    \item \textbf{Won't Have (W):} Out of scope for current release
\end{itemize}

\subsection{Example Prioritization}

\textbf{E-commerce Project:}
\begin{itemize}
    \item \textbf{Must Have:}
    \begin{itemize}
        \item User registration and login
        \item Product listing
        \item Shopping cart
        \item Checkout process
    \end{itemize}
    \item \textbf{Should Have:}
    \begin{itemize}
        \item Product search
        \item Order history
        \item Email notifications
    \end{itemize}
    \item \textbf{Could Have:}
    \begin{itemize}
        \item Product reviews
        \item Wishlist
        \item Social media login
    \end{itemize}
    \item \textbf{Won't Have:}
    \begin{itemize}
        \item Mobile app
        \item AI recommendations
    \end{itemize}
\end{itemize}

\section{Software Requirements Specification (SRS)}

\subsection{SRS Document Structure}

\begin{enumerate}
    \item \textbf{Introduction}
    \begin{itemize}
        \item Purpose
        \item Scope
        \item Definitions and acronyms
        \item References
    \end{itemize}

    \item \textbf{Overall Description}
    \begin{itemize}
        \item Product perspective
        \item Product features
        \item User classes
        \item Operating environment
        \item Constraints
        \item Assumptions
    \end{itemize}

    \item \textbf{Functional Requirements}
    \begin{itemize}
        \item Detailed requirement descriptions
        \item User stories
        \item Use cases
    \end{itemize}

    \item \textbf{Non-Functional Requirements}
    \begin{itemize}
        \item Performance
        \item Security
        \item Usability
        \item Reliability
    \end{itemize}

    \item \textbf{Appendices}
    \begin{itemize}
        \item Glossary
        \item Diagrams
        \item Mockups
    \end{itemize}
\end{enumerate}

\section{Requirements Validation}

\subsection{Validation Techniques}
\begin{itemize}
    \item \textbf{Reviews:} Stakeholders review requirements document
    \item \textbf{Prototyping:} Build mockups to validate understanding
    \item \textbf{Test Case Generation:} Write tests to verify requirements are testable
    \item \textbf{Traceability:} Ensure all requirements are linked to tests
\end{itemize}

\subsection{Good Requirements Characteristics (SMART)}
\begin{itemize}
    \item \textbf{S}pecific - Clear and unambiguous
    \item \textbf{M}easurable - Can be verified
    \item \textbf{A}chievable - Technically feasible
    \item \textbf{R}elevant - Aligned with project goals
    \item \textbf{T}ime-bound - Has deadline or priority
\end{itemize}

\section{Common Mistakes to Avoid}

\begin{enumerate}
    \item \textbf{Vague requirements:} "The system should be fast" (How fast?)
    \item \textbf{Missing stakeholders:} Not consulting all user types
    \item \textbf{Gold plating:} Adding unnecessary features
    \item \textbf{No prioritization:} Treating all requirements equally
    \item \textbf{Lack of validation:} Not confirming requirements with stakeholders
\end{enumerate}

\section{Practical Exercise}

For your project, complete the following:
\begin{enumerate}
    \item Identify all stakeholders/user types
    \item Write at least 10 user stories
    \item Prioritize using MoSCoW
    \item Write 2-3 detailed use cases
    \item Define 5 non-functional requirements
    \item Create acceptance criteria for top 5 user stories
\end{enumerate}

\section{Summary}

\begin{itemize}
    \item Requirements are the foundation of software development
    \item Use multiple techniques to gather requirements
    \item User stories capture requirements from user perspective
    \item Use cases detail system interactions
    \item Prioritize requirements using MoSCoW
    \item Validate requirements before starting development
\end{itemize}

\end{document}
