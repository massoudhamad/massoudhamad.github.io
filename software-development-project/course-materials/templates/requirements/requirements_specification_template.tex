\documentclass[12pt,a4paper]{article}
\usepackage[utf8]{inputenc}
\usepackage[T1]{fontenc}
\usepackage[margin=2.5cm]{geometry}
\usepackage{graphicx}
\usepackage{hyperref}
\usepackage{enumitem}
\usepackage{xcolor}
\usepackage{longtable}
\usepackage{fancyhdr}
\usepackage{titlesec}

\definecolor{primary}{RGB}{220, 38, 38}
\definecolor{secondary}{RGB}{248, 113, 113}

\hypersetup{
    colorlinks=true,
    linkcolor=primary,
    urlcolor=primary
}

\pagestyle{fancy}
\fancyhf{}
\fancyhead[L]{Software Development Project}
\fancyhead[R]{Requirements Specification}
\fancyfoot[C]{\thepage}

\title{\textbf{Software Requirements Specification (SRS)}\\[0.5cm]\large Template Document}
\author{State University of Zanzibar (SUZA)}
\date{BSc Computer Science}

\begin{document}

\maketitle

\section*{Document Information}
\begin{tabular}{|l|l|}
\hline
\textbf{Project Name} & [Enter Project Name] \\
\hline
\textbf{Version} & 1.0 \\
\hline
\textbf{Date} & [Enter Date] \\
\hline
\textbf{Author(s)} & [Team Members] \\
\hline
\textbf{Status} & Draft / Review / Approved \\
\hline
\end{tabular}

\tableofcontents
\newpage

\section{Introduction}

\subsection{Purpose}
[Describe the purpose of this document and the intended audience.]

\subsection{Project Scope}
[Provide a brief description of the software being specified and its purpose.]

\subsection{Definitions and Acronyms}
\begin{tabular}{|l|l|}
\hline
\textbf{Term} & \textbf{Definition} \\
\hline
SRS & Software Requirements Specification \\
\hline
UI & User Interface \\
\hline
API & Application Programming Interface \\
\hline
[Add more] & [Definition] \\
\hline
\end{tabular}

\subsection{References}
[List any documents or resources referenced.]

\section{Overall Description}

\subsection{Product Perspective}
[Describe how the software fits into the larger system or environment.]

\subsection{Product Features}
[Summarize the major features of the software.]

\subsection{User Classes and Characteristics}
[Describe the different types of users who will use this software.]

\begin{tabular}{|l|l|l|}
\hline
\textbf{User Class} & \textbf{Description} & \textbf{Technical Level} \\
\hline
Administrator & System manager & High \\
\hline
End User & Regular user & Low to Medium \\
\hline
[Add more] & [Description] & [Level] \\
\hline
\end{tabular}

\subsection{Operating Environment}
[Describe the environment in which the software will operate.]
\begin{itemize}
    \item Operating System:
    \item Browser Requirements:
    \item Hardware Requirements:
    \item Network Requirements:
\end{itemize}

\subsection{Design and Implementation Constraints}
[List any constraints that will affect the design or implementation.]

\subsection{Assumptions and Dependencies}
[List any assumptions made and external dependencies.]

\section{Functional Requirements}

\subsection{Feature 1: [Feature Name]}

\textbf{FR-001: [Requirement Title]}
\begin{itemize}
    \item \textbf{Description:} [Detailed description of the requirement]
    \item \textbf{Priority:} High / Medium / Low
    \item \textbf{Input:} [What input is required]
    \item \textbf{Processing:} [What processing occurs]
    \item \textbf{Output:} [What output is expected]
    \item \textbf{Acceptance Criteria:} [How to verify this requirement is met]
\end{itemize}

\textbf{FR-002: [Requirement Title]}
\begin{itemize}
    \item \textbf{Description:} [Detailed description]
    \item \textbf{Priority:} High / Medium / Low
    \item \textbf{Input:} [Input]
    \item \textbf{Processing:} [Processing]
    \item \textbf{Output:} [Output]
    \item \textbf{Acceptance Criteria:} [Criteria]
\end{itemize}

\subsection{Feature 2: [Feature Name]}
[Repeat the format above for each feature]

\section{Non-Functional Requirements}

\subsection{Performance Requirements}
\begin{itemize}
    \item \textbf{NFR-P01:} The system shall respond to user requests within 3 seconds.
    \item \textbf{NFR-P02:} The system shall support at least 100 concurrent users.
    \item [Add more performance requirements]
\end{itemize}

\subsection{Security Requirements}
\begin{itemize}
    \item \textbf{NFR-S01:} All passwords shall be encrypted using industry-standard encryption.
    \item \textbf{NFR-S02:} User sessions shall timeout after 30 minutes of inactivity.
    \item [Add more security requirements]
\end{itemize}

\subsection{Usability Requirements}
\begin{itemize}
    \item \textbf{NFR-U01:} The system shall be accessible on mobile devices.
    \item \textbf{NFR-U02:} New users shall be able to complete basic tasks within 5 minutes without training.
    \item [Add more usability requirements]
\end{itemize}

\subsection{Reliability Requirements}
\begin{itemize}
    \item \textbf{NFR-R01:} The system shall have 99\% uptime.
    \item \textbf{NFR-R02:} Data shall be backed up daily.
    \item [Add more reliability requirements]
\end{itemize}

\section{User Stories}

\subsection{User Story Template}
\textbf{As a} [type of user], \textbf{I want} [some goal] \textbf{so that} [some reason].

\subsection{User Story 1}
\textbf{As a} [user type], \textbf{I want} [action] \textbf{so that} [benefit].

\textbf{Acceptance Criteria:}
\begin{enumerate}
    \item Given [context], when [action], then [result]
    \item Given [context], when [action], then [result]
\end{enumerate}

\subsection{User Story 2}
[Repeat format above]

\section{Use Cases}

\subsection{Use Case 1: [Use Case Name]}
\begin{tabular}{|l|p{10cm}|}
\hline
\textbf{Use Case ID} & UC-001 \\
\hline
\textbf{Name} & [Use Case Name] \\
\hline
\textbf{Actor(s)} & [Primary Actor] \\
\hline
\textbf{Description} & [Brief description] \\
\hline
\textbf{Preconditions} & [What must be true before] \\
\hline
\textbf{Main Flow} &
\begin{enumerate}
    \item Step 1
    \item Step 2
    \item Step 3
\end{enumerate} \\
\hline
\textbf{Alternative Flows} & [Describe alternatives] \\
\hline
\textbf{Postconditions} & [What is true after] \\
\hline
\end{tabular}

\section{Requirements Traceability Matrix}

\begin{tabular}{|l|l|l|l|}
\hline
\textbf{Req ID} & \textbf{User Story} & \textbf{Test Case} & \textbf{Status} \\
\hline
FR-001 & US-001 & TC-001 & Pending \\
\hline
FR-002 & US-002 & TC-002 & Pending \\
\hline
\end{tabular}

\section{Approval}

\begin{tabular}{|l|l|l|}
\hline
\textbf{Role} & \textbf{Name} & \textbf{Signature/Date} \\
\hline
Project Supervisor & & \\
\hline
Team Lead & & \\
\hline
\end{tabular}

\end{document}
