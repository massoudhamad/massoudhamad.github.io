\documentclass[12pt,a4paper]{article}
\usepackage[utf8]{inputenc}
\usepackage[T1]{fontenc}
\usepackage[margin=2.5cm]{geometry}
\usepackage{graphicx}
\usepackage{hyperref}
\usepackage{enumitem}
\usepackage{xcolor}
\usepackage{longtable}
\usepackage{fancyhdr}
\usepackage{titlesec}

\definecolor{primary}{RGB}{220, 38, 38}
\definecolor{pass}{RGB}{34, 197, 94}
\definecolor{fail}{RGB}{239, 68, 68}
\definecolor{pending}{RGB}{234, 179, 8}

\hypersetup{
    colorlinks=true,
    linkcolor=primary,
    urlcolor=primary
}

\pagestyle{fancy}
\fancyhf{}
\fancyhead[L]{Software Development Project}
\fancyhead[R]{Test Plan Document}
\fancyfoot[C]{\thepage}

\title{\textbf{Test Plan Document}\\[0.5cm]\large Template for Quality Assurance}
\author{State University of Zanzibar (SUZA)}
\date{BSc Computer Science}

\begin{document}

\maketitle

\section*{Document Information}
\begin{tabular}{|l|l|}
\hline
\textbf{Project Name} & [Enter Project Name] \\
\hline
\textbf{Version} & 1.0 \\
\hline
\textbf{Date} & [Enter Date] \\
\hline
\textbf{Author(s)} & [Team Members] \\
\hline
\textbf{Reviewed By} & [Reviewer Name] \\
\hline
\end{tabular}

\tableofcontents
\newpage

\section{Introduction}

\subsection{Purpose}
This document describes the test plan for [Project Name]. It outlines the testing approach, test items, features to be tested, testing tasks, and resources required.

\subsection{Scope}
[Define what will and will not be tested.]

\textbf{In Scope:}
\begin{itemize}
    \item [Feature 1]
    \item [Feature 2]
    \item [Feature 3]
\end{itemize}

\textbf{Out of Scope:}
\begin{itemize}
    \item [Feature/aspect not being tested]
\end{itemize}

\subsection{Testing Objectives}
\begin{enumerate}
    \item Verify all functional requirements are implemented correctly
    \item Validate non-functional requirements (performance, security)
    \item Identify and report defects
    \item Ensure system reliability before deployment
\end{enumerate}

\section{Test Strategy}

\subsection{Testing Levels}

\subsubsection{Unit Testing}
\begin{itemize}
    \item \textbf{Objective:} Test individual components/functions
    \item \textbf{Responsibility:} Developers
    \item \textbf{Tools:} Jest, JUnit, pytest, etc.
    \item \textbf{Coverage Target:} 80\% minimum
\end{itemize}

\subsubsection{Integration Testing}
\begin{itemize}
    \item \textbf{Objective:} Test interactions between components
    \item \textbf{Responsibility:} Developers/QA
    \item \textbf{Approach:} Bottom-up / Top-down / Big Bang
\end{itemize}

\subsubsection{System Testing}
\begin{itemize}
    \item \textbf{Objective:} Test the complete integrated system
    \item \textbf{Responsibility:} QA Team
    \item \textbf{Environment:} Staging
\end{itemize}

\subsubsection{User Acceptance Testing (UAT)}
\begin{itemize}
    \item \textbf{Objective:} Validate system meets business requirements
    \item \textbf{Responsibility:} End users/Stakeholders
    \item \textbf{Environment:} UAT Environment
\end{itemize}

\subsection{Testing Types}

\begin{tabular}{|l|l|l|}
\hline
\textbf{Test Type} & \textbf{Description} & \textbf{Priority} \\
\hline
Functional Testing & Verify features work as expected & High \\
\hline
Usability Testing & Evaluate user experience & Medium \\
\hline
Performance Testing & Test response times and load & Medium \\
\hline
Security Testing & Identify vulnerabilities & High \\
\hline
Regression Testing & Ensure changes don't break existing features & High \\
\hline
\end{tabular}

\section{Test Environment}

\subsection{Hardware Requirements}
\begin{itemize}
    \item Server: [specifications]
    \item Client: [specifications]
\end{itemize}

\subsection{Software Requirements}
\begin{itemize}
    \item Operating System: [OS]
    \item Browser: Chrome, Firefox, Safari (latest versions)
    \item Database: [database name and version]
\end{itemize}

\subsection{Test Tools}
\begin{tabular}{|l|l|l|}
\hline
\textbf{Tool} & \textbf{Purpose} & \textbf{Version} \\
\hline
Jest/JUnit/pytest & Unit Testing & [version] \\
\hline
Postman & API Testing & [version] \\
\hline
Selenium & UI Automation & [version] \\
\hline
JMeter & Performance Testing & [version] \\
\hline
\end{tabular}

\section{Test Cases}

\subsection{Test Case Template}

\begin{tabular}{|l|p{10cm}|}
\hline
\textbf{Test Case ID} & TC-001 \\
\hline
\textbf{Test Title} & [Descriptive title] \\
\hline
\textbf{Module} & [Module/Feature name] \\
\hline
\textbf{Priority} & High / Medium / Low \\
\hline
\textbf{Preconditions} & [What must be true before test] \\
\hline
\textbf{Test Steps} &
\begin{enumerate}
    \item Step 1
    \item Step 2
    \item Step 3
\end{enumerate} \\
\hline
\textbf{Test Data} & [Input data] \\
\hline
\textbf{Expected Result} & [What should happen] \\
\hline
\textbf{Actual Result} & [Fill during execution] \\
\hline
\textbf{Status} & \textcolor{pending}{Pending} / \textcolor{pass}{Pass} / \textcolor{fail}{Fail} \\
\hline
\textbf{Comments} & [Any notes] \\
\hline
\end{tabular}

\subsection{Authentication Test Cases}

\textbf{TC-AUTH-001: User Registration}
\begin{tabular}{|l|p{10cm}|}
\hline
\textbf{Test Case ID} & TC-AUTH-001 \\
\hline
\textbf{Test Title} & Verify user can register with valid data \\
\hline
\textbf{Preconditions} & User is on registration page \\
\hline
\textbf{Test Steps} &
\begin{enumerate}
    \item Enter valid username
    \item Enter valid email
    \item Enter valid password
    \item Confirm password
    \item Click Register button
\end{enumerate} \\
\hline
\textbf{Expected Result} & User is registered and redirected to login page \\
\hline
\textbf{Status} & \textcolor{pending}{Pending} \\
\hline
\end{tabular}

\vspace{0.5cm}

\textbf{TC-AUTH-002: User Login}
\begin{tabular}{|l|p{10cm}|}
\hline
\textbf{Test Case ID} & TC-AUTH-002 \\
\hline
\textbf{Test Title} & Verify user can login with valid credentials \\
\hline
\textbf{Preconditions} & User is registered \\
\hline
\textbf{Test Steps} &
\begin{enumerate}
    \item Enter valid email
    \item Enter valid password
    \item Click Login button
\end{enumerate} \\
\hline
\textbf{Expected Result} & User is logged in and redirected to dashboard \\
\hline
\textbf{Status} & \textcolor{pending}{Pending} \\
\hline
\end{tabular}

\subsection{[Feature Name] Test Cases}
[Add more test cases following the template above]

\section{Bug Report Template}

\begin{tabular}{|l|p{10cm}|}
\hline
\textbf{Bug ID} & BUG-001 \\
\hline
\textbf{Title} & [Brief description] \\
\hline
\textbf{Reported By} & [Name] \\
\hline
\textbf{Date} & [Date] \\
\hline
\textbf{Severity} & Critical / High / Medium / Low \\
\hline
\textbf{Priority} & High / Medium / Low \\
\hline
\textbf{Status} & New / In Progress / Fixed / Verified / Closed \\
\hline
\textbf{Environment} & [Browser, OS, etc.] \\
\hline
\textbf{Steps to Reproduce} &
\begin{enumerate}
    \item Step 1
    \item Step 2
    \item Step 3
\end{enumerate} \\
\hline
\textbf{Expected Behavior} & [What should happen] \\
\hline
\textbf{Actual Behavior} & [What actually happens] \\
\hline
\textbf{Screenshots} & [Attach if applicable] \\
\hline
\textbf{Assigned To} & [Developer name] \\
\hline
\end{tabular}

\section{Test Schedule}

\begin{tabular}{|l|l|l|l|}
\hline
\textbf{Phase} & \textbf{Start Date} & \textbf{End Date} & \textbf{Responsible} \\
\hline
Test Planning & [date] & [date] & QA Lead \\
\hline
Test Case Design & [date] & [date] & QA Team \\
\hline
Unit Testing & [date] & [date] & Developers \\
\hline
Integration Testing & [date] & [date] & QA Team \\
\hline
System Testing & [date] & [date] & QA Team \\
\hline
UAT & [date] & [date] & Stakeholders \\
\hline
\end{tabular}

\section{Test Metrics}

\subsection{Key Metrics to Track}
\begin{itemize}
    \item Test Case Pass Rate = (Passed / Total) × 100
    \item Defect Density = Defects / Lines of Code
    \item Test Coverage = (Tested Requirements / Total Requirements) × 100
    \item Defect Leakage = Defects found after release
\end{itemize}

\subsection{Test Summary Report Template}

\begin{tabular}{|l|l|}
\hline
\textbf{Metric} & \textbf{Value} \\
\hline
Total Test Cases & [number] \\
\hline
Passed & [number] \\
\hline
Failed & [number] \\
\hline
Blocked & [number] \\
\hline
Not Executed & [number] \\
\hline
Pass Rate & [percentage]\% \\
\hline
Total Bugs Found & [number] \\
\hline
Critical Bugs & [number] \\
\hline
\end{tabular}

\section{Exit Criteria}

Testing will be considered complete when:
\begin{itemize}
    \item All planned test cases have been executed
    \item Test pass rate is at least 95\%
    \item No critical or high-severity bugs remain open
    \item All medium bugs have been reviewed and accepted
    \item UAT sign-off is obtained
\end{itemize}

\section{Risks and Mitigation}

\begin{tabular}{|l|l|l|}
\hline
\textbf{Risk} & \textbf{Impact} & \textbf{Mitigation} \\
\hline
Insufficient testing time & High & Prioritize critical tests \\
\hline
Environment unavailability & Medium & Set up backup environment \\
\hline
Incomplete requirements & High & Clarify with stakeholders early \\
\hline
\end{tabular}

\section{Approval}

\begin{tabular}{|l|l|l|}
\hline
\textbf{Role} & \textbf{Name} & \textbf{Signature/Date} \\
\hline
QA Lead & & \\
\hline
Project Supervisor & & \\
\hline
\end{tabular}

\end{document}
